\documentclass{article}
\title{Analysis report of "Design of Millimeter Wave Microstrip Reflectarrays"}
\author{Pavlo Tokariev\\Inria, I3S, Université Côte d'Azur, Sophia Antipolis}
\date{\today}

\begin{document}
    \maketitle

    \begin{enumerate}
        \item First reading: introduction and conclusion (what is it about?)
        \item Second reading: the authors are brilliant! (detect good ideas)
        \begin{enumerate}
            \item the authors had a strong motivation to perform this research: which one?
            \item they established something successful in that direction that
            \item constitutes the main contribution of the article: what exactly?
            \item what makes them/you enthusiastic about this result?
        \end{enumerate}
        \item Third reading: the article has weaknesses! (scientific doubt)
        \begin{enumerate}
            \item real scope of the experimental results?
            \item justified affirmations?
            \item exaggerated extrapolations?
            \item mostly obvious results?
        \end{enumerate}
        \item Quality:
        \begin{enumerate}
            \item of course there is this "summary aspect" with the key points of
            the article, their links with the cited references, what novelty is
            offered by the article
            \item but you should also check additional information, what is
            known elsewhere about the subject, from what sources you got
            it, and what is the reliability of these sources
            \item lastly, you should have a critical view, to evaluate the real
            scope of the article, some assertions of the article may possibly
            be a little too "optimistic." Are they some extrapolations of
            external results, possibly a little abusive? is the structure of
            the article properly made with respect to the its goal? do the
            experimental results actually support the assertions? etc.
        \end{enumerate}
        \item Scientific contribution:
        \begin{enumerate}
            \item What is the scientific domain and context of the contribution
            \item In what respect is it original w.r.t. other contemporary or past
            publications?
            \item avoid recursive readings from article references to article
            references: it rapidly goes deep in the past.
            \item rather use keywords and your ability to explore bibliography
            \item Have hindsight and do not neglect to put the article into
            context
        \end{enumerate}
        \item Writing:
        \begin{enumerate}
            \item Is the introduction informative and motivating?
            \item Are experimental material and methods properly described?
            \item In the discussion, are the main affirmations actually deducible
            from their experiments and the current knowledge?
            \item Are the results actually innovative?
            \begin{enumerate}
                \item w.r.t. the year of the article,
                \item in particular w.r.t. the previous publications of the authors.
            \end{enumerate}
        \end{enumerate}
        \item Generally:
        \begin{enumerate}
            \item within 2 to 4 pages, you cannot go into all the technical details
            \item rather have hindsight in order to understand the role of each
            technical aspect within the whole contribution
            \item one (or a few) technical aspect(s) may be a major articulation
            of the contribution; in which case you should point it out and explain why this
            aspect is of major importance
            \item being concise is also part of the exercise
        \end{enumerate}
    \end{enumerate}

    General ideas or thoughts:
    \begin{enumerate}
        \item 
    \end{enumerate}
    \bibliographystyle{plain}
    \bibliography{refs}
\end{document}
